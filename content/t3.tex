\textsc{\large Parte Experimental}
\section*{Metodología}

Elabore un diagrama que resuma la metodología desarrollada para la práctica P3.

\section*{Resultados y discusión}

Colocar en la Tabla 2 los resultados experimentales obtenidos en el laboratorio.

\begin{table}[ht!]
\caption{Resultados experimentales.}
    \centering
    \begin{tabular}{|c|c|}
    \hline
    \textbf{Material} & \textbf{Masa (g)}\\\hline
     \multirow{3}*{\shortstack{\textbf{Matraz aforado} \\[5pt] masa vacío (g):}}    &  \\\cline{2-2}
     &  \\\cline{2-2}
     &  \\\hline
     \multirow{3}*{\shortstack{\textbf{Probeta graduada} \\[5pt] masa vacía (g):}}    &  \\\cline{2-2}
     &  \\\cline{2-2}
     &  \\\hline
     \multirow{3}*{\shortstack{\textbf{Pipeta graduada} \\[5pt] masa vacío (g):}}    &  \\\cline{2-2}
     &  \\\cline{2-2}
     &  \\\hline
    \end{tabular}
\end{table}

Colocar en la Tabla 3 los resultados de la estimación de la densidad de la muestra problema.

\begin{table}[ht!]
\caption{Estimación de la densidad – muestra problema.}
    \centering
    \begin{tabular}{|c|c|}
    \hline
    \textbf{Material} & \textbf{Densidad (g/mL)}\\\hline
     \multirow{3}*{Matraz aforado}    &  \\\cline{2-2}
     &  \\\cline{2-2}
     &  \\\hline
     \multirow{3}*{Probeta graduada}    &  \\\cline{2-2}
     &  \\\cline{2-2}
     &  \\\hline
     \multirow{3}*{Pipeta graduada}    &  \\\cline{2-2}
     &  \\\cline{2-2}
     &  \\\hline
    \end{tabular}
\end{table}

\newpage

Represente gráficamente los resultados anteriores.\\[10pt]

Comente sobre la dispersión obtenida en la Figura 1.\\[10pt]

Considerando que el valor de referencia de densidad para la muestra problema es 1.119 g/mL, realizar el análisis estadístico de la densidad. Reporte los resultados en la Tabla 4.

\begin{table}[ht!]
\caption{Estimación de la densidad – muestra problema.}
    \centering
    \begin{tabular}{|c|c|c|c|}
    \hline
    \textbf{Material} & \textbf{Densidad promedio (g/mL)} & \textbf{Desv. Est. (g/mL)} & \textbf{\% DER}\\\hline
    Matraz aforado & & & \\\hline
    Probeta graduada & & & \\\hline
    Pipeta graduada & & & \\\hline
    \end{tabular}
\end{table}

Considerando los resultados obtenidos en esta práctica ¿cuál material es más exacto y cuál el más preciso? Justifique su respuesta.\\[10pt]

¿Con cuál instrumento se logra una mejor precisión?. Justifique su respuesta.